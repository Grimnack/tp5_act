\documentclass[a4paper,10pt]{report}
\usepackage[utf8]{inputenc}

% Title Page
\title{ACT - TP Heuristique}
\author{Matthieu Caron & Armand Bour}


\begin{document}
\maketitle

\section{Certificat et validation}

\paragraph{Question 1}
On décide de représenter le certificat par une liste de parts, une part serait représentée par 
un quadruplet (x,y,largeur,hauteur), avec x et y les coordonnées de la case (0,0) de la part.

\paragraph{Question 2}
On considère que dans le pire des cas chaque case du tableau pizza est une part, dans ce cas la taille du certificat
serait en $O(sizeof(int)*4*l*h)$.

\paragraph{Question 4}
La complexité de l'algorithme de solution est en $O(2*h*l)$ puisqu'on fait une matrice de bool, pour savoir si la case de la pizza est déjà prise,
et on parcours toute la matrice (pizza) en suite.

\paragraph{Question 5}
On a donc un certificat qui peut être vérifié en temps polynomial du problème, on peut donc dire que le problème de la découpe de pizza est dans NP.

\begin{abstract}
\end{abstract}

\end{document}          
